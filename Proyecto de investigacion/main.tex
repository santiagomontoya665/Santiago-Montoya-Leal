\documentclass{article}
\usepackage[utf8]{inputenc}
\usepackage[spanish]{babel}



\begin{document}


\begin{center}
\bf{\sc\Huge Paradojas matemáticas y el nacimiento de la computación}\\
\end{center}
\vspace{80pt}
\begin{center}
\bf{\sc\Huge Santiago Montoya Leal }\\
\end{center}
\vspace{80pt}
\begin{center}
\bf{\sc\Huge Universidad de Antioquia}\\
\end{center}
\vspace{80pt}
\begin{center}
\bf{\sc\Huge Facultad de ingeniería}\\
\end{center}
\vspace{20pt}
\begin{center}
\bf{\sc\Huge Ingeniería de Telecomunicaciones}\\
\end{center}
\vspace{20pt}
\begin{center}
\bf{\sc\Huge Medellin}
\end{center}
\begin{center}
\bf{\sc\Huge marzo 26 2020}\\
\end{center}\
\newpage

\section{Introducción}
\large

En el presente trabajo veremos cómo las paradojas matemáticas y la negación hacia la idea de David Hilbert de querer un sistema que  demostrara las afirmaciones verdaderas de las matemáticas llevaron al nacimiento de la computación.

\section{Paradojas matemáticas}

\large
Se puede definir la paradoja como cualquier cosa que a simple vista ser verdad pero es falso y viceversa  pareciera ser falso pero es verdadero ejemplo: cualquier cosa que existe que niega su existencia.

\vspace{10pt}
 En otras palabras la paradoja es algo verdadero que conlleva una contradicción lógica a lo que se está tomando como verdadero. (segun Tasenm pag 4)“también se emplea para hacer palpables las limitaciones que poseen las herramientas de la mente” (De Lorenzo y Fredge, 1995). Por esto mismo identificar una paradoja en cualquier área de la ciencia ya sea física, matemática, etc… puede llevar a cabo grandes avances en dichas ramas de la ciencia.

\vspace{10pt}

Estas paradojas los llevan a buscar en que aspecto más los puede afectar o beneficiar para el avanza de dicha área de la ciencia, un ejemplo de esto son “los números racionales, surgieron de una paradoja de la demostración de  que la diagonal de un cuadrado no se podía medir exactamente” (Tasenm, 2010, pág. 9).
\vspace{10pt}

\section{La idea de David Hilbert}
Debido a todas estas paradojas como la paradoja de Russel, en el siglo xx muchos de los fundamentos de la matemática fueron puestos en duda. El matemático David Hilbert quería construír un sistema que probara todas las afirmaciones verdaderas  de las matemáticas que tuviera reglas muy rigurosas. Entre ellas definiciones de la matemática pura, formulas bien descritas y que el sistema pudiera deducir por sí mismo varios teoremas efectivos(Tasenm 2010).

\vspace{10pt}

La idea no se concretó por varias, razones entre ellas que él mismo reconoció que no todos los teoremas matemáticos ni creación matemática podrían llevar a cabo dicho sistema.(Yandell 2002).

\vspace{10pt}
\section{Teorema de Kurt Gödel}
Uno de los lógicos más importantes del siglo xx Kurt Gödel fue de los primeros que no sólo dijo que dicho sistema era inalcanzable planteó los teoremas de la incompletitud en 1931 que establecen que no es posible realizar el sistema que planteaba Hilbert.
\vspace{10pt}

Gödel demostró su primer teorema de la incompletitud con una paradoja de conjuntos( si se desea ver la demostración del teorema se puede consultar en el libro de Cubo, matemático educacional de 1999 de Caudio Gutiérrez pagina 68-74).Gödel demostró que el sistema soñado por Hilbert no era posible realizarlo en ningún ámbito de la matemática ni siquiera en la suma porque era incoherente y usando la paradoja  del mentiroso aplicada a la matemática que explicó en el artículo Gödel de 1931 con una afirmación que dice que ella misma es indemostrable(Gutiérrez  1999).
\vspace{10pt}

Esto formó una polémica entre los matemáticos, aparte de que la matemáticas estaba en problemas por la crísis de los fundamentos Gödel demostró la negación a un proyecto de solución de dicha crísis.

\section{La máquina de Turing}
En 1937 Alan Turing publicó su artículo sobre los Números calculables donde se introdujo la máquina de turing y en donde se desarrolló el teorema de Gödel. Hilbert nunca dijo que cosa era el sistema que él buscaba, pero Turing le dió una definición a aquello que él buscaba y efectivamente era una máquina (Turing 1936).
Alan Turing se le considera como el padre de la teoría de la información, con su máquina que trabaja sobre una cinta infinita para hallar cálculos matemáticos que pueda realizar un humano de una forma más precisa, la máquina de por sí sola no sabrá cuando detenerse, alguien externo es que le dá un tiempo de trabajo para que la máquina se detenga.

\vspace{10pt}

Nunca se sabe cuándo un programa en la máquina de Turing se detendrá. Solamente dándole condiciones como el tiempo, como el programa no puede determinar por sí solo cuando detenerse, no es un sistema axiomático formal que buscaba realizar Hilbert(Tasenm 2010).

\vspace{10pt}
La máquina de Turing es la base de todos los computadores que se conocen al día de hoy, pues el disco duro es un muchas de máquinas de Turing juntas.
Ya después Gregori J Chaitin un matemático ruso descubrió cierta aleatoriedad de las matemáticas, gracias a ello nos aportó ideas que llevó al desarrollo de la “complejidad computacional” que junto a la máquina de Turing aportaron muchos avances a la tecnología computacional.(Tasenm 2010).
\section{Opiniones}
Se comenzaron a notar muchos avances después de la máquina de Turing un ejemplo de ello fue la maquina enigma que se utilizó mucho por los alemanes en la segunda guerra mundial y es un buen ejemplo de lo que es la criptografía. La computación fue escalando más y más de nivel siendo primera mente usados soló por empresas y gobierno y después al público en general.
\vspace{10pt}

Muchos procesos que se pueden hacer fácilmente y rápidamente tales como inscribirse a una universidad podría tardar hasta horas y horas, ahora gracias a Turing y los demás avances que vinieron después de él se hacen solo en segúndos.

\section{Conclusiones}

El sistema planteado por Hilbert no será nunca posible de desarrollar  porque por más cálculos que haga la máquina, cuando entre en una paradoja por más tiempo que pase no se detendrá, porque son problemas que nosotros mismos los creadores de dichas máquinas no podremos resolver. Las computadoras nos ayudan a resolver los problemas que sean posibles resolver, lo hacen más rápido pero con la limitación de lo que pueda calcular el ser humano.

\vspace{10pt}

Es increíble leer como las paradojas en las matemáticas traen consigo un gran avance cuando se mira la misma paradoja por diferentes ángulos ya sea de verificar de que no es posible resolverlo (que llevó al inicio de la computación), o buscar alguna manera de refutarla esto traería grandes avances. 

\vspace{10pt}

\begin{center}
\end{center}

\section{Bibliografia}
De Lorenzo, J., y Fredge, G. (1995). Grandes Matemáticos. Investigación y Ciencia. Barcelona: Temas 1.
\vspace{10pt}

Russel, B. (1976). La evolución de mi pensamiento filosófico. Madrid: Alianza Editorial.
\vspace{10pt}

Tasenm, J. S. (2010). Computacion y paradojas matematicas. Revista Digital Lampsakos,No3,pp. 9-12.
\vspace{10pt}

Yandell, B. H. (2002). The Honors Class: Hilbert's Problems and Their Solvers. New York: A K Peters.
\vspace{10pt}

Turing, A. M. (1936). On Computable Numbers, with an Application to the Entscheidungsproblem. Proceedings of the London Mathematical Society, Vol. 2 No. 42, pp. 230-265.
\vspace{10pt}

Alfonseca, M. (2000). La máquina de Turing. Números, Las matemáticas del siglo XX una mirada en 101 artículos.
\vspace{10pt}

Gutiérrez, C. (1999). El Teorema de Incompletitud de Gödel. Cubo, A Mathematical Journal, 1(1), 68-75.
\vspace{10pt}


\end{document}
